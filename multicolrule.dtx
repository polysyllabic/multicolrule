% \iffalse meta-comment
% !TEX program  = pdfLaTeX
%<*internal>
\iffalse
%</internal>
%<*readme>
multicolrule - Decorative rules between columns
=================================================

The multicolrule package lets you customize the appearance of
the vertical rule that appears between columns of multicolumn
text. It is primarily intended to work with the multicol
package, hence its name, but it also supports the twocolumn
option and \twocolumn macro provided by the standard classes
(and related classes such as the KOMA-Script equivalents).

Installation
------------

This package is provided as a .dtx file. To install it and
compile the documentation at the same time, simply process the
.dtx file with LaTeX (for example, `pdflatex multicolrule.dtx`).
To extract the package only, process the file with TeX 
(`tex multicolrule.dtx`). If you have trouble compiling the
index for the documentation, note that the documentation uses
the l3doc class to index the code, and so the makeindex command
should be the following:

`makeindex -s gind.ist -o multicolrule.ind multicolrule.idx`

This package requires LaTeX3 support to run, and compiling the
documentation requires a variety of other packages, all of which
are available on CTAN and any modern installation (TeX Live,
MiKTeX, etc.).
%</readme>
%<*internal>
\fi
\def\nameofplainTeX{plain}
\ifx\fmtname\nameofplainTeX\else
\expandafter\begingroup
\fi
%</internal>
%<*install>
\input l3docstrip.tex
\keepsilent
\askforoverwritefalse
\preamble
----------------------------------------------------------------
multicolrule --- Decorative rules between columns
Author: Karl Hagen
Email: latex@polysyllabic.com
Released under the LaTeX Project Public License v1.3c or later
See http://www.latex-project.org/lppl.txt
----------------------------------------------------------------

\endpreamble
\postamble
Copyright (C) 2018 by Karl Hagen <latex@polysyllabic.com>
 
This file may be distributed and/or modified under the
conditions of the LaTeX Project Public License (LPPL), either
version 1.3 of this license or (at your option) any later
version. The latest version of this license is in the file:
    https://www.latex-project.org/lppl.txt

This work is "maintained" (as per LPPL maintenance status) by
  Karl Hagen.

This work consists of the files multicolrule.dtx and
                                mcrule-examples.tex,
and the derived files           multicolrule.ins,
                                multicolrule.sty,
                                multicolrule.pdf, and
                                mcrule-examples.pdf.
\endpostamble
\usedir{tex/latex/multicolrule}
\generate{
  \file{\jobname.sty}{\from{\jobname.dtx}{package}}
}
%</install>
%<install>\endbatchfile
%<*internal>
\usedir{source/latex/multicolrule}
\generate{
  \file{\jobname.ins}{\from{\jobname.dtx}{install}}
}
\nopreamble\nopostamble
\usedir{doc/latex/multicolrule}
\generate{
  \file{README.md}{\from{\jobname.dtx}{readme}}
}
\ifx\fmtname\nameofplainTeX
\expandafter\endbatchfile
\else
\expandafter\endgroup
\fi
%</internal>
\RequirePackage{expl3}
\RequirePackage{xparse}
%<*driver>
\PassOptionsToPackage{svgnames}{xcolor}
\documentclass{l3doc}
\usepackage[T1]{fontenc}
\usepackage[ttscale=0.85]{libertine}
\usepackage{microtype}
\usepackage[tikz]{multicolrule}
\usepackage{longtable}
\usepackage{scrlayer-scrpage}
\usepackage{scrlayer-notecolumn}
\usepackage{bbding}
\usepackage{pgfornament}
\pagestyle{scrheadings}
\automark[subsection]{section}
\DeclareNewNoteColumn[
  reversemarginpar,
  font=\color{Navy}\footnotesize\sffamily
]{illustration}
\ProvideDocumentCommand\opt{m}{\texttt{#1}}
\NewDocumentCommand{\stydsc}{m}{^^A
  \makenote*[illustration]{^^A
    \hspace*{32pt}\parbox{\marginparwidth-32pt}{^^A
      \raggedright\texttt{#1}}^^A
  }^^A
}
\NewDocumentCommand{\mcrule}{}{\textcolor{Maroon}{multicolrule}}
\NewDocumentCommand{\kvdesc}{m}{\textcolor{Navy}{\texttt{#1}}}
\setlength{\columnsep}{24pt}
\SetMCRule{width=thin,color=Maroon,line-style=dots}
\EnableCrossrefs         
\CodelineIndex
\RecordChanges
\begin{document}
  \DocInput{\jobname.dtx}
\end{document}
%</driver>
% \fi
%
%\GetFileInfo{\jobname.sty}
%
%^^A \bool_lazy_and:nnTF and \tl_if_empty:NTF are appearing in the index even
%^^A though they're explicitly excluded below. Why is that?
%
%\DoNotIndex{\begin,
%   \bool_gset_true:N,
%   \bool_if:NTF,
%   \bool_lazy_and:nnTF,
%   \bool_new:N,\box_ht:N,\cs_generate_variant:Nn,\cs_gset:Npn,\cs_new:Npn,
%   \cs_new_nopar:Npn,\cs_set:Npn,\dim_compare_p:nNn,\dim_new:N,\dim_set:Nn,
%   \end,\ExplSyntaxOff,\ExplSyntaxOn,\fill,\hbox:n,\hspace,\ht,
%   \int_compare_p:nNn,\int_new:N,\int_set:Nn,\kern,\keys_define:nn,
%   \keys_set:nn,\msg_error:nnn,\msg_info:nnn,\msg_new:nnn,\msg_warning:nn,
%   \NewDocumentCommand,\node,\prg_replicate:nn,\ProcessKeysOptions,
%   \ProvidesExplPackage,\RequirePackage,\rule,
%   \tl_if_empty:NTF,\tl_new:N,
%   \tl_set:Nn,\vbox:n,\vbox_to_ht:nn,\vfill,\vrule,\xpatchcmd,
%   \cs_new_protected:Npn,\@ifpackageloaded,\@width}
%
%\title{^^A
%  \textsf{multicolrule} --- Decorative rules between columns\thanks{^^A
%   This file describes version \fileversion, last revised \filedate.^^A
%  }^^A
%}
%\author{^^A
%  Karl Hagen\thanks{\href{mailto://latex@polysyllabic.com}^^A
%   {latex@polysyllabic.com}}}
%\date{Released \filedate}
%
%\maketitle
%
%\changes{v0.1}{2018/12/15}{Initial public release}
%
%\begin{abstract}
% The \mcrule{} package lets you customize the appearance of the vertical rule
% that appears between columns of multicolumn text. It is primarily intended to
% work with the \pkg{multicol} package, hence its name, but it also supports
% the twocolumn option and \cs{twocolumn} macro provided by the standard
% classes (and related classes such as the KOMA-Script equivalents).
%\end{abstract}
%
%\tableofcontents
%
%\begin{multicols}{2}[\section{Introduction}]
%\SetMCRule{line-style=dashed}
%  
% In\stydsc{line-style=dashed} \LaTeX, there are two lengths that control the
% formatting between columns of multicolumn text: \cs{columnsep} specifies the
% space between adjacent columns, and \cs{columnseprule} specifies the width of
% a solid vertical rule that is placed centered between the columns. The
% \pkg{multicol} package adds the ability to change the color of the rule, but
% in both vanilla \LaTeX{} and \pkg{multicol}, the rule itself is drawn
% directly inside the routines that output the column boxes, and is therefore
% difficult for users to alter.
%
% Of course it's a legitimate question why anyone should \emph{want} to change
% this rule, or indeed use one at all, as good typography tends to avoid using
% large vertical lines.\footnote{See, for example, the remarks in the
% documentation for the \pkg{booktabs} package} In my own case, I needed to
% modify the rule because of the requirements of a particular style I was
% imitating, and having done the hard work of creating the necessary
% infrastructure for one line style, it was simple to extend the solution to a
% more general case. I hope someone else will find the options here useful.
%
% Note---in case it isn't obvious yet---that this guide illustrates the basic
% line styles that \mcrule{} makes available throught the document. The default
% line-width is 0.4pt (thin), and the default color is \texttt{Maroon}. You can
% find examples of rules created with all available options in the file
% \file{mcrule-example.pdf}.
% \end{multicols}
%
% \begin{multicols}{2}[\subsection{Bugs and Known Limitations}]
% As\stydsc{line-style=dots} this is the first release of this package, there
% are likely bugs that remain to be uncovered, as well as missing features and
% inefficient methods that should be improved upon. The development code is
% maintained on github (\url{https://github.com/polysyllabic/multicolrule}),
% and you can file feature requests or bug reports there. Alternatively, you
% can send an email to
% \href{mailto://latex@polysyllabic.com}{latex@polysyllabic.com}. I welcome
% contributions for additional styles, especially to provide more options when
% running the package without \pkg{tikz}.
%  
% The line styles that work by repeating elements in a tiled pattern may have
% significant gaps at the end of columns, particularly for larger patterns.
% You can mitigate this problem slightly by tweaking the spaces above and
% below a pattern, but the basic problem is a side-effect of the way these
% patterns are implemented (with \cs{cleaders}), which means that only an
% integer number of copies can be produced. Lines drawn with \pkg{tikz} do not
% have this problem.
% 
% I have also noticed occasional instances, most noticeably when a
% \env{multicols} environment starts near the bottom of a page and the columns
% continue to the next one, where the rules are either somewhat shorter than
% they should be or shifted upward from where they belong. In the limited
% testing I have done, this appears to be a consequence of how \pkg{multicol}
% works, as the default rules show the same behavior. I may try to nail down
% this issue in future version, but as it's an edge case that disappears when
% you add page breaks or rewrite the text to alter how the columns are filled,
% it hasn't seemed worth 
% 
% This package works by patching the output routines of either \pkg{multicol}
% or the \LaTeX{} kernel, depending on the mode of operation. It therefore will
% have no effect if you use a class or package that outputs column text via
% alternate mechanisms. This includes \pkg{parcolumns}, and probably other
% classes and packages designed to typeset parallel-column text as well, although
% I have not done a survey to determine whether this is the case. If you would
% like support for one of these, please send me an email or file a feature
% request on github and I'll see what I can do.
% 
% \mcrule{} is written using expl3 syntax, and so requires a less-than-ancient
% installation of \LaTeX. It uses the packages \pkg{l3keys2e}, \pkg{xparse},
% \pkg{xpatch}, and \pkg{xcolor}, and depending on the mode of operation may
% also require \pkg{multicol} and \pkg{tikz}. If you have an up-to-date
% distribution, these requirements should cause no issues.
% \end{multicols}
%
% \begin{multicols}{2}[\subsection{License}]
% \SetMCRule{width=ultra-thick,line-style=dotted}
% The\stydsc{\texttt{line-style=dotted, width=ultra-thick}} \mcrule{}
% package is copyright 2018 by Karl Hagen. It may be distributed and/or modified
% under the conditions of the \LaTeX{} Project Public License, either version 1.3
% of this license or (at your option) any later version. The latest version of
% this license is in \url{http://www.latex-project.org/lppl.txt} and version 1.3
% or later is part of all distributions of \LaTeX{} version 2005/12/01 or later.
%  
% This work has the LPPL maintenance status `maintained.' The Current
% Maintainer of this work is Karl Hagen.
% \end{multicols}
%
%\begin{multicols}{2}[\section{Package Options}]
% \SetMCRule{line-style=dash-dot}
%
% \subsection{Default Operation}
% If\stydsc{line-style=dash-dot} you load \mcrule{} with its default settings,
% it will enable \pkg{multicol} support, and that package will be loaded if it
% hasn't been already. Note that if you need to pass any parameters to
% \pkg{multicol}, such as |docolaction|, you should load \pkg{multicol} with
% the appropriate settings \emph{before} you load \mcrule, as \LaTeX{} does not
% support reloading packages with different parameters.
% 
% \subsection{Option `twocolumn'}
% \DescribeOption{twocolumn}
% The \mcrule{} package recognizes the option \opt{twocolumn},
% either as a package option or as a global class option. If you pass this option
% to your document class, you do not need to pass it a second time to the
% package. It is only necessary to use the package option if you plan to have a
% predominantly one-column document and use \cs{twocolumn} to switch
% temporarily into two-column mode.
% 
% Because \pkg{multicol} does not work well with the ordinary two-column
% mode, \mcrule{} is only designed to work with one or the other at a time. If
% you try to use the \opt{twocolumn} option when
% \pkg{multicol} has already been loaded, you will receive a warning, and
% nothing is guaranteed. But the custom rules will at best only appear in the
% conventional two-column mode and not within a \env{multicols} environment.
% 
% \subsection{Option `tikz'}
% \DescribeOption{tikz}
% You have access to a wider set of line styles if you also use the \pkg{tikz}
% package. Some line styles are only available if \pkg{tikz} is enabled, and
% others look better with it. The default behavior of \mcrule{} depends on the
% status of the \pkg{tikz} package at the time \mcrule{} is loaded. If
% \mcrule{} detects that \pkg{tikz} is already loaded, then tikz support will
% be enabled by default. Otherwise, you need the \opt{tikz} to enable it. This
% option also accepts explicit boolean values, so you can pass |tikz=false| if
% you want to explicitly disable tikz support. If tikz support is not enabled
% (or if it is explicitly disabled), the line styles marked \emph{tikz only}
% in section \ref{sec:linestyles} will be unavailable and errors will result
% if you try to use them.
% \end{multicols}
%
% \begin{multicols}{2}[\section{The User Interface}]
% \SetMCRule{width=2pt,line-style=circles}
% \DescribeMacro{\SetMCRule}
% The\stydsc{line-style=circles,\\width=2pt} \mcrule{} package has
% just a single user command:
%
% \cs{SetMCRule} \marg{key-value}
%
% which takes one parameter containing a key-value list of all options you
% want to set. You can issue this command in the preamble or the document body.
% Changes to the rule settings are local to the current group. For example, if
% you call \cs{SetMCRule} inside a \env{multicols} environment, the rule
% settings will revert to their previous values once the environment ends. Also
% note that any changes made with \cs{SetMCRule} when multiple columns are
% active will appear starting on the same page as your current location when you
% issue the command, and will extend the height of the full column box. It is not
% possible to have a rule change styles in the middle of a page unless you close
% out one \env{multicols} environment and begin another.
%  
% Table \ref{table:mcrulekeys} summarizes the keys available in \cs{SetMCRule}.
% The functions of each is described in detail in the sections that follow.
% \end{multicols}
%
% \begin{longtable}{lp{3.5in}}
% \caption{\cs{SetMCRule} keys\label{table:mcrulekeys}} \\
% \toprule
% Key & Purpose\\
% \midrule
% \endfirsthead
% \caption{\textbackslash SetMCRule keys (cont.)} \\
% \toprule
% Key & Purpose\\
% \midrule
% \endhead
% \kvdesc{color} & Set the color of the rule (see sec. \ref{sec:color})\\
% \kvdesc{color-model} & Set the color model of the rule (see sec.
% \ref{sec:color})\\
% \kvdesc{custom-line} & Set a custom \pkg{tikz} line for the rule 
% (\emph{tikz only}; see \ref{sec:custom})\\
% \kvdesc{custom-pattern} & Set a custom individual pattern for the rule (see
% \ref{sec:custom})\\
% \kvdesc{custom-tile} & Set a custom tiling pattern for the rule
% (see \ref{sec:custom})\\
% \kvdesc{double} & Draw two copies of the rule (see sec.
% \ref{sec:repeats})\\
% \kvdesc{line-style} & Select the type of rule printed (see sec.
% \ref{sec:linestyles})\\
% \kvdesc{single} & Draw a single copy of the rule (\emph{default};
% see sec. \ref{sec:repeats})\\
% \kvdesc{repeat} & Set the number of times to draw the rule (see
% sec. \ref{sec:repeats})\\
% \kvdesc{repeat-distance} & Set the horizontal space between
% adjacent copies of repeated rules (see sec. \ref{sec:repeats})\\
% \kvdesc{triple} & Draw three copies of the rule (see sec.
% \ref{sec:repeats})\\
% \kvdesc{width} & Set the width of the rule (see sec.
% \ref{sec:width})\\
% \bottomrule
% \end{longtable}
%
% \begin{multicols}{2}[\subsection{Styles with the `line-style' option}^^A
%   \label{sec:linestyles}]
%  \SetMCRule{width=4pt,line-style=solid-circles}
%  
% You\stydsc{line-style=solid-circles,\\width=4pt} choose a style for the rule
% with the \kvdesc{line-style} key. The default style is \kvdesc{solid}. In
% addition to the predefined styles, there are also several ways to get
% \mcrule{} to draw custom shapes in place of the column rule. The width of
% most line styles depends on the setting of \cs{columnseprule}, which is the
% default \LaTeX{} length that controls the width of the column rule
% (see section \ref{sec:width}). 
%  
% Table \ref{table:linestyles} summarizes the available line styles.
% Most of the basic shapes used to form the patterns come in three versions,
% which differ only in how closely the pattern is spaced: normal, dense, and
% loose. These settings parallel those found in \pkg{tikz}, and those
% line styles whose names are identical to the line patterns in \pkg{tikz}
% (apart from the substitution of `-' for spaces) produce the same effect.
% \end{multicols}
%
% \begin{longtable}{lp{3in}}
%   \caption{Styles available for the line-style key\label{table:linestyles}}\\
%   \toprule
%   Style & Description\\
%   \midrule
%   \endfirsthead
%   \caption{Available line-style settings (cont.)} \\
%   \toprule
%   Style & Description\\
%   \midrule
%   \endhead
%   \kvdesc{circles} & A series of hollow circles (\emph{tikz
%     only})\\
%   \kvdesc{dash-dot} & A dash followed by a square dot (\emph{tikz
%     only})\\
%   \kvdesc{dash-dot-dot} & A dash followed by two square dots
%     (\emph{tikz only})\\
%   \kvdesc{dashed} & A series of dashed lines\\
%   \kvdesc{dense-circles} & The same as \kvdesc{circles}
%     but more closely spaced (\emph{tikz only})\\
%   \kvdesc{dense-dots} & The same as \kvdesc{dots} but
%     more closely spaced\\
%   \kvdesc{dense-solid-circles} & The same as \kvdesc{solid-circles} but more closely spaced (\emph{tikz only})\\
%   \kvdesc{densely-dash-dot} & The same as \kvdesc{dash-dot} but more closely spaced (\emph{tikz only})\\
%   \kvdesc{densely-dash-dot-dot} & The same as \kvdesc{dash-dot-dot} but more closely spaced (\emph{tikz only})\\
%   \kvdesc{densely-dashed} & The same as \kvdesc{dashed} but more closely spaced\\
%   \kvdesc{densely-dotted} & The same as \kvdesc{dotted} but more closely spaced\\
%   \kvdesc{dots}  & A series of dots drawn with the period (full-stop) of the current font\\
%   \kvdesc{dotted} & A series of square dots\\
%   \kvdesc{loose-dots} & The same as \kvdesc{dots} but spaced further apart\\
%   \kvdesc{loose-circles} & The same as \kvdesc{circles} but spaced further apart (\emph{tikz only})\\
%   \kvdesc{loose-solid-circles} & The same as \kvdesc{solid-circles} but spaced further apart (\emph{tikz only})\\
%   \kvdesc{loosely-dash-dot} & The same as \kvdesc{dash-dot} but spaced further apart (\emph{tikz only})\\
%   \kvdesc{loosely-dash-dot-dot} & The same as \kvdesc{dash-dot-dot} but spaced further apart (\emph{tikz only})\\
%   \kvdesc{loosely-dashed} & The same as \kvdesc{dashed} but spaced further apart\\
%   \kvdesc{loosely-dotted} & The same as \kvdesc{dotted} but spaced further apart\\
% \kvdesc{solid} & A solid line (\emph{default})\\
% \kvdesc{solid-circles} & A series of filled circles (\emph{tikz only})\\
% \bottomrule
% \end{longtable}
%
% \begin{multicols}{2}[\subsubsection{Notes on the Styles}]
% \SetMCRule{width=thin,line-style=solid} 
% The\stydsc{line-style=solid} \kvdesc{solid}
% line style is the default. In fact, if you make no calls to \cs{SetMCRule}
% after loading \mcrule, the column divider will behave exactly as it does with
% the ordinary \pkg{multicol} package. You can alter its width and color either
% with the
% \kvdesc{width} and \kvdesc{color} keys described in
% sections \ref{sec:width} and \ref{sec:color}, respectively, or you can set the
% width directly by changing the value of \cs{columnseprule} and renewing the
% \cs{columnseprulecolor} macro. Like all line styles, the solid line can be
% repeated as many times as you like (see section \ref{sec:repeats}).
%  
% The \kvdesc{dots} style and its variants are rendered with a period
% (.) in the currently active font. This means that changing
% \cs{columnseprule} will not change the size of these dots, although, as with
% all rules, it will not appear at all if \cs{columnseprule} is set to 0pt.
%
% The \kvdesc{dotted} styles differ from \kvdesc{dots} in
% that the former are squares with side lengths equal to \cs{columnseprule}.
% This mirrors the behavior of the equivalently named dotted patterns in
% \pkg{tikz}.
% \end{multicols}
%
% \clearpage
% \subsubsection{Custom Patterns}\label{sec:custom}
%
% \begin{multicols}{2}[\noindent\texttt{custom-tile=} \marg{pattern} \marg{space above}
%   \marg{space below}]
% \SetMCRule{custom-tile={\SparkleBold}{16pt}{16pt}}
%
% There\stydsc{custom-tile= \{\textbackslash SparkleBold\} \{16pt\}\{16pt\}}
% are three options to create custom rules with \mcrule. The first is the
% \kvdesc{custom-tile} key. This creates a rule consisting of vertically
% stacked boxes of arbitrary content---the tile---running the height of the
% column separator. The \kvdesc{custom-tile} key takes three parameters,
% which must all be enclosed brackets and may not be omitted. The first
% should contain the tokens you want to appear as the content of the tile.
% The second is a dimension specifying the leading vertical space to apply
% above each copy of the tile. The third is a dimension specifying the
% trailing vertical space to insert below each copy of the tile.
%
% The rule in this section uses the \cs{SparkleBold} symbol from
% \pkg{bbding}. Notice that when you use the \kvdesc{custom-tile} parameter,
% you do \emph{not} specify a separate \kvdesc{line-style}.
% \end{multicols}
%
% \begin{multicols}{2}[\noindent\texttt{custom-pattern=} \marg{pattern} \marg{shift
%    down} \marg{shift up}]
% \SetMCRule{custom-pattern={\HandRight}{0pt}{0pt}} 
%
% The\stydsc{custom-pattern= \{\textbackslash HandRight\} \{0pt\}\{0pt\}}
% second custom option is with the \kvdesc{custom-pattern} key. The syntax
% is identical to that for \kvdesc{custom-tile}, but the content you specify
% will appear once per page or column pair (if the columns occupy less than a
% full page). This content will be vertically centered if the second and third
% parameters are both 0pt. You can shift the content down by increasing the
% second parameter, and up by increasing the third. The rule in this section
% uses the \cs{HandRight} symbol from \pkg{bbding}.
% \end{multicols}
%
% \begin{multicols}{2}[\noindent\texttt{custom-line=} \marg{draw command}]
% \SetMCRule{width=1pt,custom-line={\draw[line width=\columnseprule] (TOP) to
%   [ornament=88] (BOT);}}
%
% The\stydsc{custom-line=\{ \textbackslash draw[line width= \textbackslash
%   columnseprule] (TOP) to [ornament=88] (BOT);\},\\width=1pt} third custom
% pattern involves setting your own \pkg{tikz} drawing
% function using the key \kvdesc{custom-line}. The rule in this section is
% drawn with an ornament from \pkg{pgfornaments}. Obviously, this feature
% requires \pkg{tikz} support. The value you provide to the
% \kvdesc{custom-line} key should consist of a \pkg{tikz} command, such as
% \cs{draw}, without the surrounding \env{tikzpicture} environment.
% 
% Before the drawing command is called, \mcrule{} will set up a
% \env{tikzpicture} with both the x- and y-coordinates scaled to points, and
% two nodes, named \texttt{(TOP)} and \texttt{(BOT)}, which are set to the
% coordinates of the top and bottom of the rule. You can then specify your own
% \cs{draw} function in whatever way you like. The rule separating these
% columns was drawn with a decorative element from the \pkg{pgfornaments}
% package.
%
% This function will use the color set in
% \cs{columnseprulecolor} if you don't set it explicitly within the tikz
% command, but you must provide everything else necessary to draw the line
% correctly, including the line width. Note that this function should be
% considered experimental. It works for single-line commands such as the one
% shown in the example, but I haven't tested it with anything more elaborate. 
% \end{multicols}
%
% \begin{multicols}{2}[\subsection{Colors}\label{sec:color}]
% \SetMCRule{width=2pt,line-style=solid,color-model=cmy,color={0.7,0.5,0.3}}
% You\stydsc{line-style=solid,\\width=2pt\\color-model=cmy,\\
% color=\{0.7,0.5,0.3\}} can set colors for the rule through the
% \kvdesc{color} and, optionally, the \kvdesc{color-model} keys. \mcrule{} loads the
% \pkg{xcolor} package to manage colors, and the \kvdesc{color}
% parameter accepts any name that \pkg{xcolor} recognizes, either natively or
% as the result of any names you have defined with \cs{definecolor}
% (see the \pkg{xcolor} documentation). Note that if you want to use color names that
% are defined through the one of \pkg{xcolor}'s package options, you must load 
% \pkg{xcolor} before both \mcrule{} and \pkg{tikz} with the relevant options.
%  
% To specify a color by a numeric specification, you use the
% \kvdesc{color-model} parameter to specify any color model that
% \pkg{xcolor} recognizes (rgb, cmy, etc), and \kvdesc{color} to
% hold the color-specification list. Because that list is itself comma-separated,
% you must enclose it in brackets.
%  
% The current color setting can always be found in
% \cs{columnseprulecolor}. If you are running in twocolumn
% mode without \pkg{multicol}, this command will be provided and colors will
% work the same way they do with \pkg{multicol}. Note that setting the 
% \kvdesc{color} key causes \cs{columnseprulecolor}
% to be redefined within the current group only. If you directly redefine
% \cs{columnseprulecolor}, the color of the custom rule will
% reflect this setting. This way, the settings of any packages that might alter
% the rule color will be respected.
%\end{multicols}
%
% \begin{multicols}{2}[\subsection{Width}\label{sec:width}]
% \SetMCRule{width=thick,line-style=dash-dot-dot}
% You\stydsc{line-style=dash-dot-dot,\\width=thick} can set the width of the
% rule with the \kvdesc{width} key. Legal values are any explicit dimension or
% dimension expression, as well as with names that parallel those used by
% \pkg{tikz}, except that spaces in the key names are replaced with hyphens.
%  
% The current width of the rule is kept in \cs{columnseprule}, just as in
% vanilla \LaTeX, and if it is set separately, the custom rule's width will
% reflect this change. Note that although some line styles do not depend 
% directly on \cs{columnseprule} to calculate their actual width, the value
% of \cs{columnseprule} must be greater than 0pt for any rule to appear. This
% behavior is intentional and is in keeping with the way the default column
% rules work.
% \end{multicols}
%
% \begin{longtable}{ll}
%   \caption{Sizes of named line widths\label{table:linewidths}} \\
%   \toprule
%   Name & Width\\
%   \midrule
%   \endfirsthead
%   \caption{Sizes of named line widths (cont.)} \\
%   \toprule
%   Name & Width\\
%   \midrule
%   \endhead
%   \kvdesc{ultra-thin} & 0.1pt\\
%   \kvdesc{very-thin}  & 0.2pt\\
%   \kvdesc{thin}       & 0.4pt\\
%   \kvdesc{semithick}  & 0.6pt\\
%   \kvdesc{thick}      & 0.8pt\\
%   \kvdesc{very-thick} & 1.2pt\\
%   \kvdesc{ultra-thick} & 1.6pt\\
%   \bottomrule
% \end{longtable}
%
% \begin{multicols}{2}[\subsection{Repeated Rules}\label{sec:repeats}]
% \SetMCRule{line-style=dash-dot-dot,triple=2pt}
%  
% You can\stydsc{line-style= dash-dot-dot, triple=2pt} draw multiple,
% adjacent copies of any rule by setting the number of times to draw the rule
% with the \kvdesc{repeat} key. The space between copies is controlled with the
% \kvdesc{repeat-distance} key. Initially, this distance is set to
% \cs{columnseprule}.
%  
% The keys \kvdesc{single}, \kvdesc{double}, and \kvdesc{triple} are shorthand
% methods to set the number of repeats and the \kvdesc{repeat-distance} at the
% same time. If use the key without a value \kvdesc{repeat-distance} is set to 
% \cs{columnseprule}.
%
% There are no checks made to ensure that repeated rules will fit in the
% available space between columns, so you should be careful using these
% commands, especially with thicker rules.
% \end{multicols}
%
%\StopEventually{^^A
%  \PrintChanges
%  \clearpage
%  \PrintIndex
%}
%
% \section{Implementation}
%
%    \begin{macrocode}
%<*package>
%    \end{macrocode}
%
%    \begin{macrocode}
%<@@=mcrule>
%    \end{macrocode}
%
%\subsection{Preliminaries}
%
%    \begin{macrocode}
\ProvidesExplPackage {multicolrule} {2018/12/18} {1.0}
{Decorative~vertical~rules~between~columns}
%    \end{macrocode}
%
% We always need these packages.
%    \begin{macrocode}
\RequirePackage{l3keys2e}
\RequirePackage{xpatch}
\RequirePackage{xcolor}
%    \end{macrocode}
%
% Define the messages we use.
%    \begin{macrocode}
\msg_new:nnn {multicolrule} {patch-success} {Patched~#1.}
\msg_new:nnn {multicolrule} {patch-failure} {Error~patching~#1.}
\msg_new:nnn {multicolrule} {tikz-required}
{The~'#1'~setting~requires~tikz~to~work.~Either~load~tikz~before~you~load~
  multicolrule~or~use~multicolrule's~'tikz'~package~option.}
\msg_new:nnn {multicolrule} {multicol-loaded} {You~are~using~the~'twocolumn'~
  option~with~multicol~already~loaded.~You~will~likely~run~into~problems.}
%    \end{macrocode}
%
%\begin{variable}{\g_@@_twocolumn_bool}
% Use traditional two-column mode rather than \pkg{multicol}
%    \begin{macrocode}
\bool_new:N \g_@@_twocolumn_bool
%    \end{macrocode}
%\end{variable}
%
%\begin{variable}{\g_@@_use_tikz_bool}
% Support drawing rules with \pkg{tikz}
%    \begin{macrocode}
\bool_new:N \g_@@_use_tikz_bool
%    \end{macrocode}
%\end{variable}
%
%\begin{variable}{\l_@@_repeat_int}
% Number of times to copy the rule.
%    \begin{macrocode}
\int_new:N \l_@@_repeat_int
\int_set:Nn \l_@@_repeat_int {1}
%    \end{macrocode}
%\end{variable}
%
%\begin{variable}{\l_@@_repeat_distance_dim}
% Separation between multiple copies of the rule.
%    \begin{macrocode}
\dim_new:N \l_@@_repeat_distance_dim
%    \end{macrocode}
%\end{variable}
%
%\begin{variable}{\l_@@_color_name_tl,\l_@@_color_model_tl}
% Keep name and color model so we can set them separately while retaining
% the value of the other one.
%    \begin{macrocode}
\tl_new:N \l_@@_color_name_tl
\tl_new:N \l_@@_color_model_tl
%    \end{macrocode}
%\end{variable}
%
% If \pkg{tikz} is already loaded, enable \pkg{tikz}-sensitive line styles
% unless the user explicitly disables them. If \pkg{tikz} is not already
% loaded, these functions are disabled unless they are explicitly loaded.
%
%    \begin{macrocode}
\@ifpackageloaded{tikz}
{
  \bool_gset_true:N \g_@@_use_tikz_bool
}{}
%    \end{macrocode}
%
% Set up the keys for package options and process them.
%
%    \begin{macrocode}
\keys_define:nn {mcrule-opts}
{
  twocolumn .bool_gset:N = \g_@@_twocolumn_bool,
  tikz      .bool_gset:N = \g_@@_use_tikz_bool,
  tikz      .default:n   = true,
}
\ProcessKeysOptions{mcrule-opts}
%    \end{macrocode}
%
% \subsection{Patching Output Routines}
%
%\begin{function}{\@@_col_box:}
% Holds a reference to the box appropriate to the supported mode. We'll use
% this when we need to know the height of the columns.
%
%    \begin{macrocode}
\cs_new:Npn \@@_col_box: {}
%    \end{macrocode}
%\end{function}
%
% Now that we know what mode we're going to run in, we patch the output routine
% to substitute our custom rule for the vanilla one. Since \pkg{multicol}
% doesn't fully support twocolumn mode, we patch one or the other, but not
% both.
% 
%    \begin{macrocode}
\bool_if:NTF \g_@@_twocolumn_bool
{
  \@ifpackageloaded{multicol}
  {
    \msg_warning:nn {multicolrule} {multicol-loaded}  
  }
%    \end{macrocode}
% Provide the column-color macro from \pkg{multicol}.
%    \begin{macrocode}
  \cs_gset:Npn \columnseprulecolor {\normalcolor}
  \cs_gset:Npn \@@_col_box: {\@outputbox}
%    \end{macrocode}
% Now patch the relevant code in |\@outputdblcol|, replacing the hard-coded
% rule with a macro that we can overwrite
%    \begin{macrocode}
  \xpatchcmd{\@outputdblcol} {\normalcolor\vrule\@width\columnseprule}
  {\columnseprulecolor\mcruledivider}
  {
    \msg_info:nnn {multicolrule} {patch-success} {\@outputdblcol}
  }
  {
    \msg_info:nnn {multicolrule} {patch-failure} {\@outputdblcol}
  }
}
{
%    \end{macrocode}
%
% The patching for \pkg{multicol} is essentially the same as that for \LaTeX's
% normal |twocolumn| mode, except we ensure that \pkg{multicol} is present and
% we have two output routines to patch: one for LTR and the other for RTL
% printing.
%
%    \begin{macrocode}
  \RequirePackage{multicol}
  \cs_gset:Npn \@@_col_box: {\mult@rightbox}
  % Patch the code in multicol that creates the vertical rule.
  \xpatchcmd{\LR@column@boxes} {\vrule\@width\columnseprule} {\mcruledivider}
  {\msg_info:nnn {multicolrule} {patch-success} {\LR@column@boxes}}
  {\msg_info:nnn {multicolrule} {patch-failure} {\LR@column@boxes}}
  \xpatchcmd{\RL@column@boxes} {\vrule\@width\columnseprule} {\mcruledivider}
  {\msg_info:nnn {multicolrule} {patch-success} {\RL@column@boxes}}
  {\msg_info:nnn {multicolrule} {patch-failure} {\RL@column@boxes}}
%    \end{macrocode}
%
% Reissue \cs{LRmulticolcolumns} to update the actual code
% in \cmd{\mc@align@columns}.
%
%    \begin{macrocode}
  \LRmulticolcolumns
}
%    \end{macrocode}
%
% \subsection{Creating the Rules}
% Utility functions for different rule types
%
%\begin{macro}{\mcruledivider}
% This is the function directly called by the patched \pkg{multicol} routines.
% It's given a \LaTeX2 name so the user can redefine it if necessary. Its main
% function is to call the internal function \cs{_@@_divider:}, which contains the
% actual rule-typesetting instructions, the number of times specified in
% \cs{l_@@_repeat_int}. We only call \cs{_@@_divider:} if \cs{columnseprule} $>0$, 
% so that all line styles can be turned off by setting it to 0, just as is the
% case with the vanilla rules.
%
%    \begin{macrocode}
\cs_new:Npn \mcruledivider
{
  \bool_lazy_and:nnT
  {\dim_compare_p:nNn {\columnseprule} > {0pt}}
  {\int_compare_p:nNn {\l_@@_repeat_int} > {0}}
  {
    \_@@_divider:
    \prg_replicate:nn {\l_@@_repeat_int - 1}
    {
      \hspace{\l_@@_repeat_distance_dim}
      \_@@_divider:
    }
  }
}
%    \end{macrocode}
%\end{macro}
%
%\begin{function}{\_@@_divider:}
% This is the internal routine that contains the instructions to draw one copy
% of rule between columns. The default is identical to the original definition
% used by \pkg{multicol}. It will be reset each time the user calls
% |\MCSetRule|.
%
%    \begin{macrocode}
\cs_new:Npn \_@@_divider: {\vrule\@width\columnseprule}
%    \end{macrocode}
%\end{function}
%
%\begin{function}{\@@_pattern:nnn}
%  \begin{syntax}
%    |\__mcrule_pattern:nnn| \Arg{pattern} \Arg{space above} \Arg{space below}
%  \end{syntax}
% Typesets a single copy of a pattern, vertically centered, in a vertical
% box that is the height of the current column. The pattern must be
% something that can go in a horizontal box. The spaces above and below
% must be fixed dimensions. 
% 
%    \begin{macrocode}
\cs_new_nopar:Npn \@@_pattern:nnn #1#2#3
{
  \vbox_to_ht:nn {\box_ht:N \@@_col_box:}
  { \vfill
    \kern #2 \hbox:n{#1} \kern #3
    \vfill
  }
}
%    \end{macrocode}
%\end{function}
%
%\begin{function}{\@@_tile_pattern:nnn}
%  \begin{syntax}
%    |\__mcrule_tile_pattern:nnn| \Arg{pattern} \Arg{space above} \Arg{space below}
%  \end{syntax}
% Typesets multiple copies of pattern, tiled so as to occupy a vertical box
% that is the height of the current column. The pattern must be something
% that can go in a horizontal box. The spaces above and below must be fixed
% dimensions. 
%    \begin{macrocode}
\cs_new_nopar:Npn \@@_tile_pattern:nnn #1#2#3
{
  \vbox_to_ht:nn {\box_ht:N \@@_col_box:}
  {
    \cleaders \vbox:n {
      \kern #2 \hbox:n{#1} \kern #3
    }\vfill
  }
}
%    \end{macrocode}
%\end{function}
%
%\begin{function}{\@@_line_pattern:nnnn}
%  \begin{syntax}
%    |\__mcrule_line_pattern:nnnn| \Arg{tikz-name} \Arg{height} \Arg{space above}
%      \Arg{space below}
%  \end{syntax}
% This function can draw a line pattern using either a \pkg{tikz} name or
% directly (as a tiled pattern). The latter case is currently limited to line
% patterns that can be described in terms of a solid line of length
% \meta{height} separated by spaces above and/or below the line.
%    \begin{macrocode}
\cs_new:Npn \@@_line_pattern:nnnn #1#2#3#4
{
  \bool_if:NTF \g_@@_use_tikz_bool
  {
    \@@_pattern_line:n {#1}
  } 
  {
    \@@_tile_pattern:nnn {\rule{\columnseprule}{#2}}{#3}{#4}
  }
}
%    \end{macrocode}
%\end{function}
%
% \subsubsection{Tikz-only Routines}
%
% If we're supporting \pkg{tikz}, make sure it's loaded and redefine the
% relevant functions. We turn off |expl3| syntax to load the package because
% \pkg{tikz} relies on 2e catcodes, especially for spaces.
%
%    \begin{macrocode}
\bool_if:NTF \g_@@_use_tikz_bool
{
  \ExplSyntaxOff
  \RequirePackage{tikz}
  \ExplSyntaxOn  
%    \end{macrocode}
% 
%\begin{function}{\@@_tikz_picture:n, \@@_tikz_picture:nn}
%  \begin{syntax}
%    |\__mcrule_tikz_picture:n| \Arg{draw function}
%  \end{syntax}
% Set up the \env{tikzpicture} environment and declare two nodes, named |(TOP)|
% and |(BOT)|. This way we can pass a \cs{draw} routine directly,
% without worrying about the line's coordinates. We do a two-step call here to
% force expansion of the second argument in a way that \pkg{tikz} likes.
%
%    \begin{macrocode}
  \cs_set:Npn \@@_tikz_picture:n #1
  {
    \@@_tikz_picture:nx {#1} {\@@_col_box:}
  }
  \cs_set:Npn \@@_tikz_picture:nn #1#2
  {
    \begin{tikzpicture}[x=1pt,y=1pt,inner~sep=0pt,outer~sep=0pt]
    \node (TOP) at (0,\ht#2) {};
    \node (BOT) at (0,0) {};
    #1
    \end{tikzpicture}
  }
  \cs_generate_variant:Nn \@@_tikz_picture:nn {nx}  
%    \end{macrocode}
%\end{function}
%
%\begin{function}{\@@_pattern_line:n,\@@_pattern_line:nn}
%  \begin{syntax}
%    |\__mcrule_pattern_line:n| \Arg{tikz pattern}
%  \end{syntax}
% For the \pkg{tikz} versions of the predefined lines, we just draw
% a line the length of the column box. \meta{tikz pattern} should 
% contain the name of a line style that \pkg{tikz} recognizes.
%
%    \begin{macrocode}
  \cs_set:Npn \@@_pattern_line:n #1
  {
    \@@_pattern_line:nx {#1} {\@@_col_box:}
  }
  \cs_set:Npn \@@_pattern_line:nn #1#2
  {
    \begin{tikzpicture}[x=1pt,y=1pt,inner~sep=0pt,outer~sep=0pt]
    \draw[line~width=\columnseprule,#1] (0,\ht#2) -- (0,0);
    \end{tikzpicture}
  }
  \cs_generate_variant:Nn \@@_pattern_line:nn {nx}
%    \end{macrocode}
%\end{function}
%
%\begin{function}{\@@_circle:}
% Draw a hollow circle with a diameter equal to |\columnseprule|. This will be
% used as a tile pattern.
%
%    \begin{macrocode}
  \cs_set:Npn \@@_circle: 
  {
    \begin{tikzpicture}[x=1pt,y=1pt,inner~sep=0pt,outer~sep=0pt]
    \draw (0,0) circle[radius=.5\columnseprule];
    \end{tikzpicture}
  }
%    \end{macrocode}
%\end{function}
%
%\begin{function}{\@@_solid_circle:}
% Draw a filled circle with a diameter equal to |\columnseprule|. This will be
% used as a tile pattern.
%
%    \begin{macrocode} 
  \cs_set:Npn \@@_solid_circle: 
  {
    \begin{tikzpicture}[x=1pt,y=1pt,inner~sep=0pt,outer~sep=0pt]
    \fill (0,0) circle[radius=.5\columnseprule];
    \end{tikzpicture}
  } 
}
%    \end{macrocode}
%\end{function}
%
% In case \pkg{tikz} functions are not active, we provide stubs
% that issue error messages.
%
%    \begin{macrocode}
{
  \cs_set:Npn \@@_tikz_picture:n #1
    {\msg_error:nnn {multicolrule} {tikz-required} {#1}}
  \cs_new:Npn \@@_pattern_line:n #1
    {\msg_error:nnn {multicolrule} {tikz-required} {#1}}
  \cs_new:Npn \@@_circle: 
    {\msg_error:nnn {multicolrule} {tikz-required} {circles}}
  \cs_new:Npn \@@_solid_circle: 
    {\msg_error:nnn {multicolrule} {tikz-required} {solid-circles}}
}
%    \end{macrocode}
%
% \subsection{Color}
%\begin{function}{\_@@_set_rule_color:}
% Reset color definition in \cs{columnseprulecolor} by name or
% by model and color specification.
%
%    \begin{macrocode}
\cs_new_protected:Npn \_@@_set_rule_color:
{
  \tl_if_empty:NT \l_@@_color_name_tl
  {
    \tl_set:Nn \l_@@_color_name_tl {black}
  }
  \tl_if_empty:NTF \l_@@_color_model_tl
  {
    \cs_set:Npn \columnseprulecolor {\color{\l_@@_color_name_tl}}
  }
  {
    \cs_set:Npn \columnseprulecolor 
    {\color[\l_@@_color_model_tl]{\l_@@_color_name_tl}}
  }
}
%    \end{macrocode}
%\end{function}
%
% \subsection{Key-Values}
% Set up all the key definitions. For the line styles, this involves
% resetting \cs{_@@_divider:} to an appropriate value.
%
%    \begin{macrocode}
\keys_define:nn {mcrule}
{
  line-style .choice:,
  line-style / solid .code:n = \cs_set:Npn \_@@_divider: 
    {\vrule\@width\columnseprule},
  line-style / dots .code:n = \cs_set:Npn \_@@_divider:
    {\@@_tile_pattern:nnn {.}{1pt}{1pt}},
  line-style / dense-dots .code:n = \cs_set:Npn \_@@_divider: 
    {\@@_tile_pattern:nnn {.}{1pt}{0pt}},
  line-style / loose-dots .code:n = \cs_set:Npn \_@@_divider:
    {\@@_tile_pattern:nnn {.}{2pt}{2pt}},
  line-style / circles .code:n = \cs_set:Npn \_@@_divider: 
    {\@@_tile_pattern:nnn {\@@_circle:}{1pt}{1pt}},
  line-style / dense-circles .code:n = \cs_set:Npn \_@@_divider:
    {\@@_tile_pattern:nnn {\@@_circle:}{1pt}{0pt}},
  line-style / loose-circles .code:n = \cs_set:Npn \_@@_divider: 
    {\@@_tile_pattern:nnn {\@@_circle:}{2pt}{2pt}},
  line-style / solid-circles .code:n = \cs_set:Npn \_@@_divider: 
    {\@@_tile_pattern:nnn {\@@_solid_circle:}{1pt}{1pt}},
  line-style / dense-solid-circles .code:n = \cs_set:Npn \_@@_divider:
    {\@@_tile_pattern:nnn {\@@_solid_circle:}{1pt}{0pt}},
  line-style / loose-solid-circles .code:n = \cs_set:Npn \_@@_divider:
    {\@@_tile_pattern:nnn {\@@_solid_circle:}{2pt}{2pt}},
  line-style / dotted .code:n = \cs_set:Npn \_@@_divider: 
    {\@@_line_pattern:nnnn {dotted}{\columnseprule}{1pt}{1pt}},
  line-style / densely-dotted .code:n = \cs_set:Npn \_@@_divider:
    {\@@_line_pattern:nnnn {densely~dotted}{\columnseprule}{1pt}{0pt}},
  line-style / loosely-dotted .code:n = \cs_set:Npn \_@@_divider:
    {\@@_line_pattern:nnnn {loosely~dotted}{\columnseprule}{2pt}{2pt}},
  line-style / dashed .code:n = \cs_set:Npn \_@@_divider:
    {\@@_line_pattern:nnnn {dashed}{3pt}{1.5pt}{1.5pt}},
  line-style / densely-dashed .code:n = \cs_set:Npn \_@@_divider:
    {\@@_line_pattern:nnnn {densely~dashed}{3pt}{1pt}{1pt}},
  line-style / loosely-dashed .code:n = \cs_set:Npn \_@@_divider: 
    {\@@_line_pattern:nnnn {loosely~dashed}{3pt}{3pt}{3pt}},
  line-style / dash-dot .code:n =
    \cs_set:Npn \_@@_divider: {\@@_pattern_line:n{dash~dot}},
  line-style / densely-dash-dot .code:n =
    \cs_set:Npn \_@@_divider: {\@@_pattern_line:n{densely~dash~dot}},
  line-style / loosely-dash-dot .code:n =
    \cs_set:Npn \_@@_divider: {\@@_pattern_line:n{loosely~dash~dot}},
  line-style / dash-dot-dot .code:n =
    \cs_set:Npn \_@@_divider: {\@@_pattern_line:n{dash~dot~dot}},
  line-style / densely-dash-dot-dot .code:n =
    \cs_set:Npn \_@@_divider: {\@@_pattern_line:n{densely~dash~dot~dot}},
  line-style / loosely-dash-dot-dot .code:n =
    \cs_set:Npn \_@@_divider: {\@@_pattern_line:n{loosely~dash~dot~dot}},
  color .code:n = {
    \tl_set:Nn \l_@@_color_name_tl {#1}
    \_@@_set_rule_color:
  },
  color-model .code:n = {
    \tl_set:Nn \l_@@_color_model_tl {#1}
    \_@@_set_rule_color:
  },
  custom-line .code:n = \cs_set:Npn \_@@_divider:
    {\@@_tikz_picture:n {#1}},
  custom-pattern .code:n = \cs_set:Npn \_@@_divider:
    {\@@_pattern:nnn #1},
  custom-tile .code:n = \cs_set:Npn \_@@_divider:
    {\@@_tile_pattern:nnn #1},
  width .choice:,
  width / ultra-thin  .code:n = \dim_set:Nn \columnseprule {0.1pt},
  width / very-thin   .code:n = \dim_set:Nn \columnseprule {0.2pt},
  width / thin        .code:n = \dim_set:Nn \columnseprule {0.4pt},
  width / semithick   .code:n = \dim_set:Nn \columnseprule {0.6pt},
  width / thick       .code:n = \dim_set:Nn \columnseprule {0.8pt},
  width / very-thick  .code:n = \dim_set:Nn \columnseprule {1.2pt},
  width / ultra-thick .code:n = \dim_set:Nn \columnseprule {1.6pt},
  width / unknown     .code:n = {\dim_set:Nn \columnseprule {#1}},
  repeat              .int_set:N = \l_@@_repeat_int,
  repeat-distance     .dim_set:N = \l_@@_repeat_distance_dim,
  single              .meta:n = {
    repeat = 1,
    repeat-distance = #1
  },
  single              .default:n = \columnseprule,
  double              .meta:n = {
    repeat = 2,
    repeat-distance = #1
  },
  double              .default:n = \columnseprule,
  triple              .meta:n = {
    repeat = 3,
    repeat-distance = #1
  },
  triple              .default:n = \columnseprule,
}
%    \end{macrocode}
%
% \subsection{User Interface}
% With only one command, this section is short. All we do is set whatever keys
% the user passes. All the real work is done in the definitions above.
%
%\begin{macro}{\SetMCRule}
%\begin{syntax}
%  |\SetMCRule| \marg{key-value list}
%\end{syntax}
%    \begin{macrocode}
\NewDocumentCommand{\SetMCRule}{m}
{
  \keys_set:nn {mcrule} {#1}
}
%    \end{macrocode}
%\end{macro}
%
%\Finale